\chapter{Graphene.js: Building Reusuable Web Components for Data Visualization}

\section{Purpose}

\section{Motivation}

\subsection{Shortcomings of current methods}

Cytoscape Web (\url{http://cytoscapeweb.cytoscape.org/})

Cytoscape.js (\url{http://cytoscape.github.io/cytoscape.js}) is a graph drawing library and a potential choice for network visualization. It contains an impressive HTML5 Canvas based rendering engine, but currently lacks the ability to customize the following:

\begin{itemize}
\item Color gradients for nodes
\item Marker ends and node shapes beyond a predefined set
\item Control points for Bezier curves
\item Placement of edge start and end points
\end{itemize}

Furthermore, since Cytoscape.js produces a canvas drawing and not DOM elements, the graph elements are unable to be used with other JavaScript libraries (for example a popover plugin that acts on each node) nor bound to custom events.

\section{Solution}

Graphene.js (\url{https://github.com/stanleygu/graphene}) addresses these issues by using a different approach. 
Graphene creates interactive diagrams through two-way-binding of user provided data objects to a customizable SVG template.
Thus, no custom rendering engine is required as it is performed by the browser, and the rich SVG vocabulary (which may be created through a text editor or through a graphical SVG drawing program) may be used to define nearly any type of edge, arrow marker, or Bezier curve.
Graphene templates may also be customized for visualizations beyond node-edge graphs, such as charts and animations.

\subsection{Modular Web Components}

\subsubsection{SVG vs. HTML Canvas}

\subsection{Single Page Web Applications}

declarative programming should be used for building user interfaces and wiring software components, while imperative programming is excellent for expressing business logic


Decouple DOM manipulation from application logic. This improves the testability of the code. Regard application testing as equal in importance to application writing. Testing difficulty is dramatically affected by the way the code is structured.
Decouple the client side of an application from the server side. This allows development work to progress in parallel, and allows for reuse of both sides.
Guide developers through the entire journey of building an application: from designing the UI, through writing the business logic, to testing.
Angular follows the MVC pattern of software engineering and encourages loose coupling between presentation, data, and logic components. Using dependency injection, Angular brings traditional server-side services, such as view-dependent controllers, to client-side web applications. Consequently, much of the burden on the backend is reduced, leading to much lighter web applications.

\subsection{Data-Binding}


\section{Implementation}
\subsection{Data-Binding}
\subsubsection{Angular.js}
\subsection{SVG Manipulation}
\subsubsection{D3.js}

\section{Summary}
