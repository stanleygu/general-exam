%
% ----- copyright and title pages
%
\Title{Carbon.js: A Modern Toolkit for Building Web Applications for Systems Biology}
\Author{Stanley Gu}
\Year{2014}
\Program{UW Bioengineering}

\Chair{Herbert M. Sauro}{Associate Professor}{Department of Bioengineering}
\Signature{James B. Bassingthwaighte}
\Signature{Daniel L. Cook}
\Signature{Thomas Hazlet}
\Signature{Rick James}

\copyrightpage

% \titlepage  

% --- sample stuff only -----
% unusual footnote not found in a real thesis
% You just use the \titlepage as commented out above

{\Degreetext{General Examination \\
  submitted in partial fulfillment of the\\ requirements for the degree of}
 \def\thefootnote{\fnsymbol{footnote}}
 \let\footnoterule\relax
 \titlepage
 }
\setcounter{footnote}{0}

% --- end-of-sample-stuff ---
 
%
% ----- signature and quoteslip are gone
%

%
% ----- abstract
%


\setcounter{page}{-1}
\abstract{%

This sample dissertation is an aid to students who are attempting
to format their theses with \LaTeX, a sophisticated
text formatter widely available at the University of Washington
and other institutions of higher learning.
 
\begin{itemize}
\item It describes the use of a specialized
macro package developed specifically for thesis production
at the University.
The macros customize \LaTeX\ for the correct thesis style,
allowing the student to concentrate on the substance of
his or her text.%
\footnote{See Appendix A to obtain the source to this
 thesis and the style file.}
\item It demonstrates the solutions to a variety of
formatting challenges found in thesis production.
\item It serves as a template for a real dissertation.
\end{itemize}
}
 
%
% ----- contents & etc.
%
\tableofcontents
\listoffigures
%\listoftables  % I have no tables
 
%
% ----- glossary 
%
\chapter*{Glossary}      % starred form omits the `chapter x'
\addcontentsline{toc}{chapter}{Glossary}
\thispagestyle{plain}
%
\begin{glossary}
\item[AJAX] (Asynchronous Javascript and XML) is a group of web development techniques used for creating asynchronous web applications.
\item[Angular.js] is client-side MVC framework, maintained by Google and an open source community, that assists in creating single-page web applications.
\item[D3.js] is a JavaScript library that uses data-binding to drive the creation and control of dynamic and interactive visualizations in web browsers.
\item[Docker] is an open-source project that automates the deployment of applications inside Linux Containers.
  Linux Containers utilize the underlying Linux kernel for isolating an application's view of the operative environment without the need for starting any virtual machines.
\item[DOM] (Document Object Model) is a language independent convention for representing and interacting with objects in HTML, XHTML, and XML documents.
\item[DOMAM] (DOM as Data Model) is a design pattern used my often by jQuery-based web applications.
\item[Express.js] is a Node.js based web application framework.
\item[Flask] is a lightweight web application framework written in Python.
\item[HTML] is the standard markup language used to create web pages.
\item[HTML5] is the fifth revision of the HTML standard that introduces markup and APIs for complex web applications.
\item[JavaScript] is a dynamic programming language that is commonly used in client-side scripts in web browsers and also server-side programming through Node.js.
\item[Node.js] is a software platform, written in JavaScript, for scalable server-side and networking applications.
\item[Python] is a widely used general-purpose, high-level programming language.
\item[SVG] is an XML-based vector image format for two-dimensional graphics with support for interactivity and animation.

\end{glossary}
 
%
% ----- acknowledgments
%
\acknowledgments{% \vskip2pc
  % {\narrower\noindent
  The author wishes to express sincere appreciation to the
  University of Washington and the Department of Bioengineering.
  % \par}
}

%
% ----- dedication
%
\dedication{\begin{center}

%to my dear wife, Joanna

\end{center}}
