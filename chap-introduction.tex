\chapter{Introduction}

\section{Purpose}
The purpose of 
This chapter serves to set the stage for the relevance of this body of work by exploring some challenges facing computational systems biology and how new web based tools can help.

* Present a toolkit for to make building systems biology web applications easier
* Give a preliminary demonstration at how a new web tool could look like and how it may open new possibilities for a new breed of app

\section{Motivation}

Computational Systems Biology is a growing inter-discplinary field that focuses on understanding complex biological systems. \autocite{kitano2002computational}
Traditionally, biologists would build scientific hypotheses on paper and then test them through \textit{in vitro} and \textit{in vivo} experimentation, now this work is assisted by computer software.
Systems biology is the progression of classical molecular biology from a qualitative to a quantitative science.
Software can help not only data collection and analysis, but formulate and test hypotheses through virtual experimentation.
Computational models described here are formal descriptions of a biomedical process, and in particular intracellular kinetic models described by ordinary differential equations.
A simulation is a dynamic realization of the model that describes the evolution of the process from a particular set of starting conditions, which may be tested experimentally.
Simulation experiments allow research in areas that were once difficult to approach by providing information that is difficult or impossible to measure. \autocite{edwards2001silico}
Thus, progress in Systems Biology can result in accelerated drug discovery, safer medicines, and improved understanding of pathogenesis. \autocite{kitano2010grand, mack2004can}

In other research-oriented industries where software and simulation has been integrated, such as automobile, aerospace, and telecommunication, have seen immense cost savings and increases in efficiency.
Virtual cars are "driven" and virtual aircrafts "flown" under simulated conditions before production and manufacture. \autocite{ghosh2010connecting}
Evidence suggests that the Systems Biology software and modeling tools still has some more room to grow in order to realize its full potential (for example, the pharmaceutical industry spends 25\% of revenues on research and development, almost twice that of any other R\&D industry). \autocite{economist2005models}

So what are some of the challenges in modeling?
It may be a difficult task for some researchers to write models, as they are usually phrased in terms of differential or stochastic equations.
Once completed, models may be so far removed from the biological context that is hard to identify what the model is about or what assumptions were made.
Thus, these models are difficult to search for by a computer and are not readily accessible to the community.
Simulations have generate a large amount of data.
However, when the underlying simulation procedure and model are not available, these results are difficult to interpret.

The Systems Biology community is actively addressing these concerns through the development of standards.
The range of standards include the description of computational models \autocite{hucka:2002d, LloydCellML2004}, annotation of the model assumptions and constituents \autocite{novere2005minimum}, annotation of simulation experiments \autocite{kohn2008sed}, and annotation of simulation data \autocite{dada2010sbrml}.
However, these model standards and ontologies are abstract and meant primarily to be understood by computer scientists and software developers.
It is the role of software applications to help mitigate this kind of complexity.
A modeler should not have to be concerned with all these technical details and be allowed to focus on creating and analyzing the model.

Thus, it is the goal of this work to increase researcher productivity by bridge the gap between modeling standards and end users.

\subsection{Why the Web?}

The bulk of the work described in this writing is centered around the web browser as the primary user interface, so it is worth discussing why the web was chosen as the target platform.
Due to not having any experience with graphical software development at the start of this dissertation project, this author surveyed the various software platforms available to invest in.
In order to make an application widely available to researchers, writing cross platform applications is a technical challenge, especially for research groups with limited software engineering resources. \autocite{cusumano1999netscape}
Cisco estimates that the number of devices connected to the Internet will swell from about 10 billion today to 50 billion by 2020 \autocite{clark2014internet}
With the advent of modern browsers, HTML5 based web applications are cross-platform, off-line capable, and perpetually updated. \autocite{o2007web}

By virtue of expecting of expecting users to have a network connection, web applications may also be more social.
A recent study \autocite{dabbish2012social} suggests that social networks can increase productivity.
Dabbish et al. surveyed users of the open software repository GitHub, and found that transparency and collaboration through a web interface promotes user innovation, knowledge sharing, and community building.

Scientific web applications may also change the way models and results are made available to the scientific community.
Models are often discovered through journal publications, which contain a text description of the model, static tables and charts of the results, and possibly a supplementary section with data and the encoded model description file in standard format (such as SBML or CellML).
With the rise of online journals, the viewing application is often the web browser.
The tools developed in this dissertation can allow the model and reproducible simulation results directly embedded or linked into an interactive figure in the article itself.

\section{Specific Results}
\subsection{Aim 1 - Chapter~\ref{chap:graphene}: Produce a web library for building interactive and graphical applications}
\subsection{Aim 2 - Chapters~\ref{chap:tidal},~\ref{chap:redox}: Integrate Graphene into new and existing applications}
\subsection{Aim 3 - Chapters~\ref{chap:carbon},~\ref{chap:engine}: Build server-side architecture for modeling applications}

