\chapter{Web Applications}
\label{appendix-web}
\section{A brief history of web applications}
\label{appendix-web-history}

\autocite{w3c2014history}
\autocite{berners2014design}
\begin{itemize}
  \item 1980: Tim Berners-Lee, at CERN, made ENQUIRE a system to use and share documents
  \item 1991: HTML publicly released
  \item 1995: JavaScript written in 10 days by Brendan Eich to be released with Netscape Navigator 2.0
  \item 1996: CSS
  \item 1997: ECMAScript 1
  \item 1998: ECMAScript 2
  \item 1999: ECMAScript 3
  \item 2005: AJAX \autocite{garrett2005ajax}
  \item 2006: jQuery released
  \item 2008: JavaScript the Good Parts 176 pages, JavaScript the Definitive Guide 1100 pages
  \item 2009: Node.js, Angular.js, ECMAScript 5
  \item 2010-today: Backbone, Knockout.js, Ember, Handlebars.js, etc\ldots
\end{itemize}

\subsection{Problem with Web Security}

\autocite{grier2008secure}

Meanwhile, the Internet and World Wide Web \autocite{berners2000weaving} has become perhaps the most powerful medium for information sharing. \autocite{bollacker1998citeseer, wilkinson2003motivations, page1999pagerank}
Modern web applications have many advantages over traditional desktop applications.
Software developers typically face a challenge in deploying software applications to their clients (Figure~\ref{fig:deployment-problem}).
End users may use a wide variety of platforms, such as Apple OS X, Microsoft Windows, and Linux for desktop computers and Apple iOS, Google Android, and Microsoft Windows Phone for mobile devices.
Each platform has a different pattern for building native binaries, which requires special expertise and is an obstacle particularly for smaller development teams and research labs.
Web applications are inherently cross platform, available on all devices with a capable web browser (Figure~\ref{fig:web-app-deployment}).
Updates are also delivered immediately, which is typically not the case for desktop softwares.
For these reasons, HTML5 web applications are a compelling platform for biomodeling applications as well.

\begin{figure}
  \centering
  \includegraphics[width=\textwidth,natwidth=610,natheight=642]{images/deployment-problem.png}
  \caption{Software developers often face a challenge in deploying their code base to end users due to the widespread use of multiple platforms.}
  \label{fig:deployment-problem}
\end{figure}
\begin{figure}
  \centering
  \includegraphics[width=\textwidth,natwidth=610,natheight=642]{images/web-app-deployment.png}
  \caption{Web applications solve the cross-platform problem.}
  \label{fig:web-app-deployment}
\end{figure}

\subsection{DOM}

The Document Object Model (DOM) is one of the central conventions in web browsers for representing

\begin{figure}
  \centering
  \includegraphics[width=0.5\textwidth, page=22, trim=0cm 0cm 11cm 0cm, clip=true]{images/Figures.pdf}
  \caption{Hierarchy of objects in a web browser DOM. Like HTML, DOM elements follow a tree structure.}
  \label{Figure:dom}
\end{figure}


\subsection{Single Page Web Applications}

\begin{figure}
  \centering
  \includegraphics[width=\textwidth, page=3]{images/Figures.pdf}
  \caption{How JQuery and Angular scale for complex JavaScript Web Applications.}
  \label{Figure:jquery-vs-angular}
\end{figure}

declarative programming should be used for building user interfaces and wiring software components, while imperative programming is excellent for expressing business logic


Decouple DOM manipulation from application logic. This improves the testability of the code. Regard application testing as equal in importance to application writing. Testing difficulty is dramatically affected by the way the code is structured.
Decouple the client side of an application from the server side. This allows development work to progress in parallel, and allows for reuse of both sides.
Guide developers through the entire journey of building an application: from designing the UI, through writing the business logic, to testing.
Angular follows the MVC pattern of software engineering and encourages loose coupling between presentation, data, and logic components. Using dependency injection, Angular brings traditional server-side services, such as view-dependent controllers, to client-side web applications. Consequently, much of the burden on the backend is reduced, leading to much lighter web applications.

\subsection{HTML5}
\label{sec:html5}
